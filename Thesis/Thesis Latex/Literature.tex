%%% LyX 2.0.5.1 created this file.  For more info, see http://www.lyx.org/.
%%% Do not edit unless you really know what you are doing.
%\documentclass[twoside,english]{report}
%\usepackage[T1]{fontenc}
%\usepackage[latin9]{inputenc}
%\setcounter{secnumdepth}{3}
%\setcounter{tocdepth}{3}
%\usepackage[active]{srcltx}
%\usepackage{verbatim}
%\usepackage{graphicx}
%\usepackage{setspace}
%\usepackage[numbers]{natbib}
%\doublespacing
%
%\makeatletter
%
%%%%%%%%%%%%%%%%%%%%%%%%%%%%%%% LyX specific LaTeX commands.
%\providecommand{\LyX}{L\kern-.1667em\lower.25em\hbox{Y}\kern-.125emX\@}
%%% Because html converters don't know tabularnewline
%\providecommand{\tabularnewline}{\\}
%
%\makeatother
%
%\usepackage{babel}
%\begin{document}

\chapter{Literature Review}
Automated recognition of text has been an active subject of research since the early days of computers. A 1972 survey cites nearly 130 works on the subject \cite{harmon1972automatic}. Despite the age of the subject, it remains one of the most challenging and exciting areas of research in computer science. In recent years it has grown into a mature discipline, producing a huge body of work.
{\color{blue}Printed Arabic text is like handwritten Latin text, such that connection of characters is an inherent property for Arabic script whether it is typed, printed or handwritten. Most of errors and deficiencies of Arabic recognition systems comes from the segmentation stages Various segmentation algorithms have been proposed in the literature. Given the vast number of papers published on OCR, it is impossible to include all the segmentation methods in this survey. (Zidouri, Sarfraz, Shahab, \& Jafri, 2005)}

\emph{Need to decide about what to write in this section. Should it contain literature review about every stage in the system? i.e. Pre-processing, segmentation, letters recognition, and post-processing.}

\section{Arabic On-line Recognition Approaches}


\section{Arabic Text Segmentation}
\emph{copy paste from the paper}

\section{English On-line handwriting Recognition}

%\end{document}
