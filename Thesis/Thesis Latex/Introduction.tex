%% LyX 2.0.5.1 created this file.  For more info, see http://www.lyx.org/.
%% Do not edit unless you really know what you are doing.
%\documentclass[12pt,english]{report}
%\usepackage{mathptmx}
%\renewcommand{\familydefault}{\rmdefault}
%\usepackage[T1]{fontenc}
%\usepackage[latin9]{inputenc}
%\usepackage[a4paper]{geometry}
%\setcounter{secnumdepth}{2} % Changed from 3 to 2. 0-chapter 1-section 2-subsection 
%\setcounter{tocdepth}{2} % Changed from 3 to 2. 0-chapter 1-section 2-subsection 
%\setlength{\parskip}{\medskipamount}
%\setlength{\parindent}{0pt}
%\usepackage{verbatim}
%\usepackage{pdfpages}
%\usepackage{graphicx}
%\usepackage{subfig} %% This package has to be here
%\usepackage{setspace}
%\usepackage{arabtex}
%\usepackage[numbers]{natbib}
%\usepackage{nomencl}
%\usepackage{amsthm}
%\usepackage{amsmath}
%\usepackage{amsfonts}
%\usepackage{etoolbox}
%\newtoggle{edit-mode}
%\toggletrue{edit-mode}  
%%%\toggletrue{edit-mode}
%\iftoggle{edit-mode}{
%\geometry{verbose,tmargin=2cm,bmargin=2cm,lmargin=2cm,rmargin=6cm,headheight=1cm,headsep=1cm,footskip=1cm, marginparwidth=5cm}
%}{
%\geometry{verbose,tmargin=2cm,bmargin=2cm,lmargin=2cm,rmargin=2cm,headheight=1cm,headsep=1cm,footskip=1cm}
%}
%
%\makenomenclature
%
%%% Theorem Styles
%\newtheorem{theorem}{Theorem}[section]
%%% Definition Styles
%\theoremstyle{definition}
%\newtheorem{definition}{Definition}[section]
%\newtheorem{example}{Example}[section]
%\theoremstyle{remark}
%\newtheorem{remark}{Remark}
%
%\usepackage[linesnumbered]{algorithm2e}
%
%\begin{document}
%\printnomenclature{}
%
%\tableofcontents{}

%Instructions - How to write an introduction:
%-------------------------------------------------
%You can't write a good introduction until you know what the body of the paper says. Consider writing the introductory section(s) after you have completed the rest of the paper, rather than before.
%Be sure to include a hook at the beginning of the introduction. This is a statement of something sufficiently interesting to motivate your reader to read the rest of the paper, it is an important/interesting scientific problem that your paper either solves or addresses. You should draw the reader in and make them want to read the rest of the paper.
%
%The next paragraphs in the introduction should cite previous research in this area. It should cite those who had the idea or ideas first, and should also cite those who have done the most recent and relevant work. You should then go on to explain why more work was necessary (your work, of course.) 
% 
%What else belongs in the introductory section(s) of your paper? 
%A statement of the goal of the paper: why the study was undertaken, or why the paper was written. Do not repeat the abstract. 
%Sufficient background information to allow the reader to understand the context and significance of the question you are trying to address. 
%Proper acknowledgement of the previous work on which you are building. Sufficient references such that a reader could, by going to the library, achieve a sophisticated understanding of the context and significance of the question. 
%The introduction should be focused on the thesis question(s).  All cited work should be directly relevent to the goals of the thesis.  This is not a place to summarize everything you have ever read on a subject.
%Explain the scope of your work, what will and will not be included. 
%A verbal "road map" or verbal "table of contents" guiding the reader to what lies ahead. 
%Is it obvious where introductory material ("old stuff") ends and your contribution ("new stuff") begins? 
%Remember that this is not a review paper. We are looking for original work and interpretation/analysis by you. Break up the introduction section into logical segments by using subheads.
 
\chapter{Introduction}

\section{Background}

\subsection{Handwriting Recognition}

\nomenclature{$HWR$}{Handwriting recognition}
\iftoggle{edit-mode}{\hspace{0pt}\marginpar{HWR Motivation 1 - handwriting importance and survival}}{}
Writing, which has made much of the culture and civilization possible, was developed a long time ago as a mean to expand human memory and to facilitate communication. 
Despite the long standing prediction that digital computers will challenge its future, handwriting persists. 
Nowadays, handwriting remains a commonly used mean of communication and recording of information in the daily life, and therefore, a growing interest in the \emph{handwriting character recognition} (HWR) field has emerged in recent years, and has now been a topic of research for over four decades .
Producing documents were hugely simplified by computers, however, the pen and paper are still the natural medium for many important tasks such as notes taking in class.

Converting handwritten text into its digital equivalent is highly motivated by the ease and convenience of the digital representation.
Not only this is useful for making digital copies of handwritten documents, but also in many automated processing tasks, such as searching, indexing, automatic mail sorting, processing, sharing and more \cite{noaparast2009persian}.

\iftoggle{edit-mode}{\hspace{0pt}\marginpar{HWR Motivation 2 - Keyboard-less devices}}{}
Other motivation for recognizing handwritten scripts is the rapid transition from personal computers and laptops to the usage of keyboard-less smart-phone and table devices that which are too small to have convenient keyboards thus requiring pen or finger gestures to enter data \cite{connell2000online}. 

\nomenclature{$OCR$}{Optical character recognition}
\iftoggle{edit-mode}{\hspace{0pt}\marginpar{HWR as OCR}}{} 
HWR was defined by Plamondon and Srihari as "the task of transforming a language represented in its spatial form of graphical marks into its symbolic representation" \cite{plamondon2000online}.
HWR is a special case of emph{optical character recognition} (OCR), an important field in pattern recognition that is defined as the science of electronically converting a scanned, photoed or sensed of both typewritten or printed texts into machine-encoded text. OCR has been steadily evolving during its history. It has always been a favorite testing ground for new ideas in pattern recognition giving rise to an exciting set of research topics and producing many powerful practical applications.
However, since many experiments of new ideas in pattern recognition were conducted on isolated characters, the results are not always immediately reflected in OCR applications \cite{burrow2004arabic}  .
OCR is considered one of the best applications of machine vision and one of the most successful research branches in pattern recognition theory. 
Although considered a well developed technological field, OCR remains an area of active scientific research and creative engineering \cite{borovikov2004survey}.
Besides recognition, handwriting is associated with other types of analysis, such as, signature verification, writer identification, etc.

\iftoggle{edit-mode}{\hspace{0pt}\marginpar{Levels of difficulties in HWR}}{}
There are different types with varying complexity of problems within HWR, which is based on how the data is presented to the recognition system, at what level the data can be unambiguously broke into pieces (e.g. individual characters or words), and the transcription complexity of the language used \cite{bahlmann2005advanced}. In one extreme there is the case of isolated characters written inside graphical boxes in which the segmentation problem is already solved. The opposite extreme is the case of cursive unrestricted handwriting in which words, or portion of a word, is written with a single stroke, i.e, ligatures connect adjacent letters. Indeed, HWR systems have a strong history in making use of this graduation in difficulty \cite{bahlmann2005advanced}. 

\iftoggle{edit-mode}{\hspace{0pt}\marginpar{The cursive nature of the Arabic handwriting}}{}
Each script has a set of icons that have certain basic shapes. These are known as characters or letters.
Every script has it's own rules on how letters are combined to form shapes that represent higher level linguistic units \cite{plamondon2000online}.
Unlike the Latin script, in which non-cursive handwriting, i.e., using isolated characters, is possible and common, the Arabic script, the only case is the cursive in both handwritten and printed text.
One problem in unconstrained HWR is the huge variety in the writing of different people. 
Another extreme difficulty task is the so-called segmentation task, that is, the process of dividing an entire writing into sub-units (typically characters) \cite{bahlmann2005advanced}.

%%%%%%%%%%%%%%%%%%%%%%%%%%%%%%%%%%%%%%%%%%%%%%
\subsection{Off-line versus On-line Handwriting Recognition}

\iftoggle{edit-mode}{\hspace{0pt}\marginpar{Introduction}}{}
The field of HWR can be classified in several ways. However, the most common categorization is the one that distinguish between \emph{off-line} (also called static) and \emph{on-line} (also called dynamic).
Off-line HWR focuses on documents that have been written on paper at some previous point of time. A digital image of the document is fed to the computer, and the system attempts to convert the spatial representation of the letters into digital symbols \cite{al2011online}. 
In contrast, on-line HWR is performed on a digital representation of the text written on a special digitizer, tablet or smart-phone device, where sensors pick up the pen-tip movements and the two-dimensional coordinates of successive points of the writing as a function of time are stored.
Figure \ref{fig:offline_vs_online} shows typical input signals that can be analyzed in both cases.

\iftoggle{edit-mode}{\hspace{0pt}\marginpar{Similarities and advantages of on-line and off-line}}{}
Off-line HWR techniques can be applied on on-line data by constructing a static image of the on-line script. 
However, it has been shown that the information about the pen dynamics can be used to obtain a better recognition accuracies than the static data alone. 
In the other direction, the success of on-line systems makes it attractive to consider developing off-line systems that first estimate the trajectory of the writing from off-line data and then use on-line HWR techniques. 
However, the difficulty of reconstructing the temporal data has led to few such systems so far \cite{plamondon2000online}.
An advantage of on-line handwritten data over off-line data is the availability of the stroke segmentation and the order of writing. 
Ink in static images must first be separated from the image background, creating a potential source of error. 
The ability to detect the states of "pen-down" (when the pen touches the tablet or the finger touches the touch screen) and "pen-up" can also be used. 

\iftoggle{edit-mode}{\hspace{0pt}\marginpar{State of the art HWR}}{}
In general, off-line HWR systems are less accurate than on-line systems, but, they are now good enough that they have a significant economic impact on for specialized domains such as interpreting handwritten postal addresses on envelopes and reading courtesy amounts on bank checks.

\begin{figure}
	\centering
        \subfloat[]{
            \label{fig:offline}
            \includegraphics[width=0.5\textwidth]{./figures/offline}
        }
        \subfloat[]{
           \label{fig:online}
           \includegraphics[width=0.5\textwidth]{./figures/online}
        }        
    \caption{(a) Off-line word. The image of the word is converted into gray-level pixels using a scanner. (b) On-line word. The x; y coordinates of the pen-tip are recorded as a function of time with a digitizer \cite{plamondon2000online}.}
   \label{fig:offline_vs_online}
\end{figure}
 
\nomenclature{$WP$}{Word parts}
\iftoggle{edit-mode}{\hspace{0pt}\marginpar{A general flow for HWR - off-line and on-line}}{}
There are large variety of approaches for HWR, and one may find it very difficult to draw a general scheme that outlines the general flow for this task. However, there are several stages that can be found in most approaches.
The HWR general scheme consist of the following. 
The difference between the methods for on-line and off-line handwriting recognition, while having much in common the difference in the data representation may impose different challenges in each case and thus the most effort can be spent on different stages.
In the first stage, the preprocessing stage, usually consists of stages that include filtering of noises and normalization. 
Then, depending on the nature of the system, the script may undergoes segmentation into basic units that may be words, WPs, single characters or graphemes.
Then, a feature extraction techniques are applied and then labeled by the classier. 
In many cases, a post-processing stage is applied in which the language model is used to search for the most likely string in the lexicon.
The off-line case usually additional stages that consists of layout analysis and extraction of text lines, are usually applied. 
Each text line is then segmented into isolated basic unit images. 
The whole process is straightforward for well printed or well written documents \cite{saba2010survey}. 
However, in the case of historical or badly printed document much effort is invested in a preprocessing stage that include smoothing, writing flow reconstruction, purification, and more. 
In the on-line case, the stylos motion is sampled at equal time intervals and pass the information to the recognition algorithm. The samples goes through preprocessing stage that include filtering of noises, re-sampling the stroke information to obtain equidistant stroke, and normalized to a standard size. Additional preprocessing may include slant and slope corrections. 

%%%%%%%%%%%%%%%%%%%%%%%%%%%%%%%%%%%%%%%%%%%%%%

\subsection{The Holistic versus the Analytic Approach}

\iftoggle{edit-mode}{\hspace{0pt}\marginpar{problems imposed by the open vocabulary}}{}
While there exist a wide variety of approaches to cursive script recognition, research in this field has established two main approaches, one is the analytic approach \cite{abdulla2008off, sari2002off, dinges2011offine, elanwar2012unconstrained}, and the other is the holistic approach \cite{biadsy2011segmentation}. 

\iftoggle{edit-mode}{\hspace{0pt}\marginpar{The analytic approach}}{}
The analytic approach involves segmentation of the input curve (that represents a word) into basic units (usually letters) and classification of each part of the text into individual characters, which are recognized and then assembled to identify the written word. 
The advantage of this approach is that it requires to maintain only a small set of trained models - one for each letter shape - to handle large vocabulary. however, the absence of consistent baselines, large variations in writing styles, and seamless connection between letters (connection is done with almost no ligatures)make segmentation into individual letters almost impossible. \cite{saabni2009hierarchical}

\iftoggle{edit-mode}{\hspace{0pt}\marginpar{The holistic approach}}{}
The holistic approach (also called a global approach) considers the global properties of the written text and recognizes the input word shape as a whole. 
Most popular methods among this group are based on analysis of the number and order of ascenders, descenders, loops and vertical strokes. 
They often rely on heavy dictionary searching that is costly and prone to be mislead by spelling errors. \cite{brodowska2011oversegmentation}
While it avoids the error-prone segmentation process the holistic approach, the recognition system needs to be trained over all words in the dictionary and to maintain and train models for each word. 
It may be possible for small vocabulary of words, however, this is not feasible for large vocabularies (20,000 words or more). 
Since each word is constructed from a subset of the character alphabet, it is much more efficient to classify words using the analytic approach \cite{elanwar2012unconstrained}.\\

%%%%%%%%%%%%%%%%%%%%%%%%%%%%%%%%%%%%%%%%%%%%%%%%%%%%%%%

\subsubsection{Closed Vocabulary vs. Open Vocabulary}

\iftoggle{edit-mode}{\hspace{0pt}\marginpar{Importance of the dictionary size}}{}
The vocabulary, from which the words in the test set are taken, has a major impact on how difficult the HWR task is.

\iftoggle{edit-mode}{\hspace{0pt}\marginpar{Closed and open vocabulary definitions}}{}
Closed-vocabulary HWR system is capable of recognizing words from a predetermined limited size dictionary. 
The restricted vocabulary set is usually called a lexicon.
There are no established definitions for the lexicon size, however, the following terms are usually used:
\begin{itemize}
\item small lexicon - tens of words.
\item medium lexicon - hundreds of words.
\item large lexicon - thousands of words.
\item very large lexicon - tens of thousands of words.
\end{itemize}
Open-vocabulary tasks refer to recognition of any words without the constraint of being in a dictionary.

\iftoggle{edit-mode}{\hspace{0pt}\marginpar{Recognition difficulty}}{}
Closed-vocabulary tasks are easier than open-vocabulary ones because only certain subsequences of letters are possible when limited by a dictionary.
The vocabulary size is a common constraint of current recognition systems. 
Most current HWR systems are able to handle small and medium size lexicons.

\iftoggle{edit-mode}{\hspace{0pt}\marginpar{problems imposed by the open vocabulary}}{}
The lexicon is a key-point post-processing stage in many systems, because the linguistic knowledge helps to filter out many possible options that are not included in the lexicon, and consequently raise the recognition rate.
The adhesion to a some limited dictionary, may also limit the computational complexity. 
Although most research efforts have been devoted to closed vocabulary systems, open vocabulary systems has also been proposed.
However, their accuracy is still far below those relying on a small vocabulary \cite{koerich2003large, shu1996line}.

%%%%%%%%%%%%%%%%%%%%%%%%%%%%%%%%%%%%%%%%%%%%%%%%%%%%%%%

\subsection{Arabic Handwriting Recognition}

\iftoggle{edit-mode}{\hspace{0pt}\marginpar{The Arabic spread}}{}
The Arabic script is one of the descendants of the Aramaic script. 
The earliest known document written using the Arabic script is dating from 512 AD.  
The Arabic language is spoken, as their first language, by nearly 350 million people around the world \cite{zeki2011segmentation}, and written by more than 100 million people, in over 20 different countries.
This makes it one of the five most common languages in the world and one of the six United Nations official languages since 1974 \cite{burrow2004arabic}. 
Although Arabic is used mainly in the Arab countries, which is about 5.5\% of the world population, almost all Muslims, around 25\% of the world population can read Arabic script as it is the language of the Holy Qur'an \cite{zeki2011segmentation}.

\iftoggle{edit-mode}{\hspace{0pt}\marginpar{The Arabic Alphabet usage in other languages}}{}
The use of Arabic language extended in the 7th and 8th centuries from India to the Atlantic ocean due to the Islamic conquests \cite{saabni2009efficient}. 
Consequently, more than twenty different languages adopted the Arabic alphabet with some changes. 
Examples of such languages are Farsi, Urdu, Malay, Housa and Ottoman Turkish.
However, it must be mentioned that some of those languages are currently using Latin characters, but in general, people can still read the Arabic script.

\iftoggle{edit-mode}{\hspace{0pt}\marginpar{Literary vs. daily language}}{}
Although the spoken Arabic is different from country to country, the written Arabic is standard system used all over the Arab world.
The literary language is called Modern Standard Arabic or Literary Arabic is which it is currently the only official form of Arabic, used in most written documents as well as in formal spoken occasions, such as lectures and news broadcasts. 

\iftoggle{edit-mode}{\hspace{0pt}\marginpar{The growing interest in the Arabic HWR}}{}
Despite the fact that Arabic alphabets are used in many languages, Arabic Character Recognition has not received enough interests from researchers, and therefore, little research progress has been achieved as compared to the one done on Latin or Chinese. 
It has almost only started in 1975 by Nazif \cite{nazif1975system}, while the earlier research efforts in Latin may be traced back to the middle of the 1940s.
However, recent years have shown a considerable increase in the number of research papers related to Arabic character recognition.
The challenging nature of handwriting recognition and segmentation has attracted the attention of researchers from industry and academic circles \cite{al2010development}.
Nevertheless, Arabic HWR, in both on-line and off-line, is still remains an open problem \cite{zeki2011segmentation}.

%%%%%%%%%%%%%%%%%%%%%%%%%%%%%%%%%%%%%%%%%%%%%%%%%%%%%%%

\subsubsection{Characteristic of the Arabic writing system}

\iftoggle{edit-mode}{\hspace{0pt}\marginpar{Basic properties}}{}
Arabic script is written from right to left in a semi-cursive manner in both printed and handwritten.
It consists of 28 basic letters, 12 additional special letters, and 8 diacritics. \emph{TODO: and a figure}
Most letters are written in four different letter shapes depending on their position in a word, e.g., the letter \RL{`}  (Ain) appears as \RL{`} (isolated), \RL{`-}(initial),\RL{-`-} (medial) and \RL{-`} (final). 
Among the basic letters, six are Dis-connective - \RL{A} (Alef), \RL{d} (Dal), \RL{_d} (Thal), \RL{r} (Reh), \RL{z} (Zain) and \RL{w} (Waw). 
Dis-connective letters do not connect to the following letter and have only two shapes each, isolated and final. 
The presence of these letters interrupts the continuity of the graphic form of a word, and closes the WP. 
We denote these connected graphic parts of the word as as word parts (WPs). 
If the word-part is composed of only one letter, this letter will be in its isolated shape \cite{biadsy2011segmentation}. 

\iftoggle{edit-mode}{\hspace{0pt}\marginpar{Delayed strokes}}{}
Certain characteristics relating to the obligatory dots and strokes of the Arabic script distinguish it from Latin script, making the recognition of words in Arabic more difficult than in Roman script. 
These dots and strokes are called \emph{delayed strokes} since they are usually drawn last in the in handwritten WP/word. 
We will distinguish between two types of delayed strokes, \emph{i'jam} (\RL{A`jAm}) and \emph{harakat} (\RL{.hrkAt}). 
The old Arabic was written without dots or diacritics. 
These additional strokes that were added to the Armaic letters, were first introduced by Yahya bin Ya'mur and Nasr bin Asim around the 7th century, to prevent the Qur'an from being misread by Muslims \cite{burrow2004arabic}.

\iftoggle{edit-mode}{\hspace{0pt}\marginpar{I'jam}}{}
The i'jam are the pointing diacritics added to the main body of the letter (called rasm) and their role is to distinguish between various constants. 
For example, the medial form letters \RL{-b-} /b/, \RL{-t-} /t/, \RL{-_t-} /th/, \RL{-n-} /n/, \RL{-y-} /y/.
Typically, i'jam are not considered diacritics but part of the letter and consists of one or more dots and lines added above under or inside the letter.
Eliminating, adding or moving a i'jam produces a completely different letter and as a result a completely different word, thus, they are not omitted in the written documents or in the spoken language.
Not only dots are used as i'jam, the \RL{'} (hamza) is another type of i'jam that distinguish between the letters \RL{k} /k/ and \RL{l} /l/ in their isolated and final forms.

\iftoggle{edit-mode}{\hspace{0pt}\marginpar{Harakat}}{}
Small marks, called harakat, are used to indicate short vowels, often heard and are sometimes written when using the standard Arabic.
These diacritics are small markings used above or below the letters of a word to specify the exact pronunciation of the word.
The harakat are used in the holy book Qur'an and are commonly used in teaching material and poetry but are seldom used in day-to-day communication and handwriting neither are in use in the scientific, and business communication.

An example of a fully vocalised Arabic from the Qur'an (Al-Fatiha 1:1):

\begin{center}
\fullvocalize
\transtrue
\begin{RLtext}
bismi al-ll_ahi al-rra.hm_ani al-rra.hImi
\end{RLtext}
In the Name of All'ah, the Most Gracious, the Most Merciful...
\end{center}

\iftoggle{edit-mode}{\hspace{0pt}\marginpar{Word parts count}}{}    
Saabni and El-sana have explored a large collection of Arabic texts and extracted 300,000 different word combinations of 82,000 different WP. 
Ignoring the additional strokes reduced the number of different WP to 40,000 \cite{saabni2009efficient}. 

\iftoggle{edit-mode}{\hspace{0pt}\marginpar{Limitation of this work}}{}  
In our work we recognize and classify the main body of the letter and ignore the additional stroke entirely. 
As a result, the number of different letters drops from 29 to 18.
It is important to note that taking the delayed strokes into consideration may be exploited to boost the classification rate.

\iftoggle{edit-mode}{\hspace{0pt}\marginpar{Challenges of the Arabic language}}{}  
The main body of most Arabic letters is written by a single stroke. However, there are some letters that usually written using 2 strokes, such as the letter \RL{-k-}  which is the middle form of the letter \RL{k} (Ka). The writer usually writes \RL{-l-} and adds the final upper slanted line when the main body is completed, as if he writes an additional stroke.
Another problem arises when trying to recognize Arabic transcript, is that, different writers may write the main body of the same word part in a different number of strokes. 
For instance, the main body of the word part \RL{byt} (Bayt - Home), is usually written in a single stroke however, sometimes it may be written by some writers using 3 strokes.
We have also considered the common combination of the letter \RL{l} followed by the vowel \RL{A} as a single letter which is commonly drawn as \RL{lA}. 
Also, three consequent appearances of the letter \RL{-b-} (B) in the middle of the word-part looks as follows: \RL{-bbb-}. 
As can be seen very similar to the \RL{s} (S) letter in its medial position \RL{-s-}, the only to distinguish between the two options is by looking at the additional strokes.
For these mentioned complexities, when recognizing Arabic scripts, many researches have preferred the holistic approach. 

%%%%%%%%%%%%%%%%%%%%%%%%%%%%%%%%%%%%%%%%%%%%%%%%%%%%%%%
\subsection{Arabic script Segmentation}

Character segmentation is the process of splitting the handwritten script into basic units, usually corresponding to single letters or alternatively to different basic units usually named graphems. Graphemes are ... and they could be letters, couples of letters or part of a letter.

\iftoggle{edit-mode}{\hspace{0pt}\marginpar{Importance}}{}
Segmentation is one of the required steps in numerous off-line and on-line cursive handwritten word recognition solutions and has long been a critical area of the OCR process. The problems of segmentation persist today. .
Its importance is derived from the fact that improperly extracted characters are usually difficult to recognize correctly. 
Wrong segmentation will often results in major contribution to the error of the recognition algorithm. 
An integral part of the hand writing recognition process, when the analytic approach is considered, is segmentation. \cite{casey1996survey, brodowska2011oversegmentation}

\iftoggle{edit-mode}{\hspace{0pt}\marginpar{The context dependent of segmentation}}{}
{the segmentation decision is not a local decision, independent of previous and subsequent
decisions. Producing a good match to a library symbol is necessary, but not sufficient, for reliable
recognition. That is, a poor match on a later pattern can cast doubt on the correctness of the current
segmentation/recognition result. Even a series of satisfactory pattern matches can be judged incorrect if
contextual requirements on the system output are not satisfied. For example, the letter sequence "cl" can
often closely resemble a "d", but usually such a choice will not constitute a contextually valid result.}

\iftoggle{edit-mode}{\hspace{0pt}\marginpar{The segmentation - recognition catch-22}}{}
{It is believed, good segmentation is one reason for high accuracy character recognition. However, the task of segmentation, while usually trivial to a human, is a very challenging pattern recognition problem.
The Catch 22 in segmentation is as follows: character is a pattern that
resembles one of the symbols the system is designed to recognize. But to determine such a resemblance
the pattern must be segmented from the document image. Each stage depends on the other, and in complex
cases it is paradoxical to seek a pattern that will match a member of the system's recognition alphabet
of symbols without incorporating detailed knowledge of the structure of those symbols into the process.
\cite{casey1996survey}}

\iftoggle{edit-mode}{\hspace{0pt}\marginpar{Cursive English segmentation - techniques and achievements}}{}
There are many known method of performing cursive letters segmentation. 
The broad range of significant real world applications and the considerable difficulty of the task both contribute to the fact that there are many known methods of performing cursive word recognition - one suited better for some applications and some for the others \cite{brodowska2011oversegmentation}.
However, a correct segmentation rate is not easily achievable.

\iftoggle{edit-mode}{\hspace{0pt}\marginpar{The three approaches for segmentation}}{}
In a comprehensive survey done in \cite{casey1996survey}, the authors has pinpointed three elemental strategies for off-line cursive text segmentation in addition to many approaches that are a combinations of these three. They have noted that much of the literature on segmentation reports methods that can be characterized as a blend of these three mentioned methods.

\iftoggle{edit-mode}{\hspace{0pt}\marginpar{Dissection}}{}
The classical approach which is usually named \emph{dissection}. 
This approach attempts to segment the text into character using "character-like" properties. 
Examples of such properties are height, width, separation from neighboring components, disposition along a baseline, etc.
Conventional approaches can segment typical words accurately. 
In this regard, different strategies such as projection profile, bounding box or contour tracing exhibit promising results. 
However, they might lead to incorrect segmentation when deal with touching characters \cite{saba2010survey}.


\iftoggle{edit-mode}{\hspace{0pt}\marginpar{Recognition based segmentation}}{}
The catch 22 described above explains the fact that character segmentation and character classification steps are not totally separate. Degree of connection between them vary from a solution to a solution.
The second approach is recognition-based segmentation, in which the system searches for sub-components in the cursive that match letters in its alphabet. 
In this method, the recognition confidence, heavily affects the overall segmentation accuracy.
The general approach is to split words into segments that should be characters, pass each segment to a classifier and, if the classification results are not satisfactory (e.g. some measure of belief requirements is not met), call segmentation once more with the feedback information about rejecting the previous result. \cite{brodowska2011oversegmentation}

\iftoggle{edit-mode}{\hspace{0pt}\marginpar{The holistic approach}}{}
The holistic methods, in which the system seeks to recognize words as a whole, thus avoiding the segmentation process which was previously discussed in Section \ref{[]}.

\iftoggle{edit-mode}{\hspace{0pt}\marginpar{Over-segmentation}}{}
The most common methods of character segmentation is initial over - segmentation, i.e, finding some set of potential splitting points in the graphical representation of the word and then attempting to eliminate the improper ones
The segmentation task of some cursive and unconstrained nature of some languages such Arabic, anneals the segmentation task. 
The technique requires an over-segmentation stage which partition handwritten words into primitives that may then be processed further to provide the best segmentation.

\iftoggle{edit-mode}{\hspace{0pt}\marginpar{Issues with segmentation}}{}
In languages that are commonly written non-cursively, segmentation techniques may be used in HWR systems.
also may use segmentation techniques since human handwriting 
touching characters are error prone in segmentation stage that contributes to recognition errors.

A through survey on character segmentation is given in \cite{casey1996survey} and \cite{brodowska2011oversegmentation}.  

%%%%%%%%%%%%%%%%%%%%%%%%%%%%%%%%%%%%%%%%%%%%%%%%%%%%%%%
\newpage{}
%%%%%%%%%%%%%%%%%%%%%%%%%%%%%%%%%%%%%%%%%%%%%%%%%%%%%%%

\section{Related Work}
\label{sec:related_work}

Randa et al. \cite{elanwar2012unconstrained} proposed a two stage on-line Arabic handwritten text segmentation system based on Hidden Markov Model (HMM). 
In the first stage, SPs were nominated, and then, in a second stage, the nominated points were validated using a rules-based engine. 
The system was tested using a self-collected database named OHASD.

Digness et al. \cite{Dinges2011} used a segmentation-based recognition approach based on dividing the word into smaller pieces, which were afterwards segmented into candidate letters, and then classified into letter classes, using statistical and structural features. 
The $k$-NN classifier was used to obtain the final recognition.

A segmentation-based recognition method that operates on the stroke level for on-line Arabic handwritten words recognition was proposed by Daifallah et al. \cite{daifallah2009recognition}. 
SPs were nominated and then selected by locating semi-horizontal lines moving from right to left. 
A portion of the SPs is filtered out by applying a certain set of rules. 
Then, HMM is used to classify the sub-strokes to letters using the Hu feature. 
The letters candidates and their scoring were used to determine the best set of SPs.\\

%%%%%%%%%%%%%%%%%%%%%%%%%%%%%%%%%%%%%%%%%%%%%%%%%%%%%%%
\newpage{}
%%%%%%%%%%%%%%%%%%%%%%%%%%%%%%%%%%%%%%%%%%%%%%%%%%%%%%%

\section{Scope and Overview}

\iftoggle{edit-mode}{\hspace{0pt}\marginpar{Scope}}{}
The cursiveness of the Arabic script, prima facie, requires delaying the launch of the recognition process until the completion of the word scribing. 
However, in this study, we question the necessity of the requirement by demonstrating the feasibility of approximating the position of the segmentation points (SPs) while the stroke is being written; and by doing so, accelerating the recognition process. 
Furthermore, the obtained information can be used to significantly reduce the potential dictionary size and facilitate a later holistic recognition process.
The proposed method is an analytic approach for handwriting recognition, i.e., it involves segmentation and classification of each part of the text. 
The system operate on the stroke level enabling analysis task related to segmentation and recognition of the stroke into letters to be done while the stroke is being written.


The emphasis was on finding workable segmentation technique, rather than on the recognition rate.
The system consists of two main module, one that handles the segmentation and the other handles recognition. 

The system general flow is seen below:
\begin{figure}
\centering
\includegraphics[width=0.8\textwidth]{./figures/system_flow}
\caption{High level visualization of the system flow.}
\label{fig:system_flow}
\end{figure}

The main contributions of the thesis are as follows:
1. create a letters database from the ADAB Database
2. propose a fast trajectory recognition systems
3. propose an an ongoing recognition technique

%%%%%%%%%%%%%%%%%%%%%%%%%%%%%%%%%%%%%%%%%%%%%%%%%%%%%%%
\newpage{}
%%%%%%%%%%%%%%%%%%%%%%%%%%%%%%%%%%%%%%%%%%%%%%%%%%%%%%%

\section{Thesis Outline}
The organization of the thesis is as follows. 
Chapter \ref{} presents the letters extraction process performed on the ADAB in order to obtain letters samples.
Our method for fast Arabic letters classifier is discussed in Chapter \ref{}. 
Chapter \ref{} shows how strokes are segmentation is nominated while the stroke is being scribed, and the final segmentation is selected. 
Finally, a summary and future work are proposed in Chapter \ref{}. 

%\bibliographystyle{plainnat}
%\bibliography{references}
%\end{document}

%TODOs:
%----------
%\begin{enumerate}
%\item See the section "Overview of Arabic Letters" in \cite{abandah2009analysis}. It talks about letters frequencies and special 
%\item Talk about the WP and give statistics of WP length as described in {abandah2009analysis}
%\item Add a table containing the letters in all positions
%\item See \cite{al2011online}.
%\item See \cite{abandah2009analysis}
%\item Review \cite{bahlmann2005advanced} thesis
%\item TODO:see the in introduction of \cite{abandah2009analysis}
%\item See \cite{lorigo2006offline} - motivation section
%\end{enumerate}


%The full list is given in table \ref{table:same_rasm_letters} 
%
%\begin{table}[h]
%\begin{tabular}{ | c | c |}
%\hline
%Letter Set & Positions of similarity\\
%\hline                 
%  \RL{b}, \RL{t}, \RL{_t} & All Positions\\ 
%  \hline
%  \RL{`}, \RL{.g} & All Positions\\
%  \hline
%  \RL{.h}, \RL{j}, \RL{x} & All Positions\\
%  \hline
%  \RL{f}, \RL{q} & All Positions (very slight differences\\ 
%   					&in Isolated mode, the valley of the letter \RL{q} is deeper)\\
%  \hline
%  \RL{r}, \RL{z} & Isolated\\
%  \hline
%  \RL{h}, \RL{T} & Isolated\\
%  \hline
%\end{tabular}
%\centering
%\label{table:same_rasm_letters} 
%\caption{Arabic letters with the same rasm that are diffrentiated by the I'jam}
%\end{table}